第一部において、「序論」が「問題設定とそれに対する答え方の構え」を示すものである一方で、この「緒言」は「第一部が書かれた時代の知の配置の中で、第一部がどのような位置づけの考察なのか」を示すものだ。第一部は、誰に何をもたらすのか。そして、それを可能にするために何をするのか。

第一部がまず念頭においている読者は、私たちを駆り立てるべくますます騒がしさを増してきつつある世の喧騒に対してうんざりしつつも、実際のところどのように生きるべきかについて思いめぐらせている人々だ。そうした人々にとって、第一部は静寂を取り戻す助けとなることだろう。第一部は、人々が混乱の中に沈むことを防ぐ小舟のような存在となるに違いない。その小舟の上からは、世の喧騒が如何に簡単な無理解に立脚したものであるかがすぐに分かるようになる。そして読者は、第一部を読むことで、本質的に未解決な問題については地道な探究を重ねるしかなく、そこに魔法のような解決があることは稀なのだと、腹の底から納得できるようになるだろう。

このようにして得られる静寂において、読者の前にはこの宇宙の中で、私たちが如何に小さな存在であるかが分かるようになる。そこでは宇宙が、古の人間が思い描いたであろうような意味に満ちたものではなく、茫漠とした無根拠と無意味として現れてくる。

しかし、それにもかかわらず、読者はそこでただ単にすべてから切り離された根無し草になるわけではない。読者はそこで、自身の身体に出会う。その身体とは、自身がそこでそのように感じ、そう欲する以外の仕方では欲することができないような「固着した飢え」である。

飢えは救いではないが、飢えを無視することは苦しみを生む。固着した飢えという立脚点を得て、その渇きを癒すために、幸運ならば、読者は第一部という知識の小舟から無知が渦巻く宇宙という大海へと最後は飛び込むだろう。第一部が提供する知識の体系は、その際に自身の身を混沌から守る一助にもなるだろう。第一部は、そのようなものだ。

そのような議論を、第一部はどのようにして可能にするのか。それは、第一部は複数の専門分野で発見された知識を相互に束縛させあい、接続させあうことによってだ。そうすることで、第一部は複数の専門分野を一つに繋ぎ合わせ、統合的な知の道筋を浮かび上がらせる。その際に、第一部は各専門分野の基本的な知識しか用いない。それは、第一部が示す道筋が、実は現代人の眼前に当然のごとく在り続けていたにもかかわらず、ほとんど誰にも顧みられなかったものだからだ。この意味で、第一部は特定の専門分野に直接寄与するものではないし、文学的なテクストによって人々の情動を揺さぶるものでもない。それらは第一部の目的ではないからだ。同様に、特定の専門分野への寄与や文学的なテクストを期待する読者に、第一部が与えられるものはない。それにもかかわらず、第一部が示す道筋は、現代人の知識を整理することによって、様々な専門分野や文学に従事する人々を含む幅広い層に寄与するだろう。
