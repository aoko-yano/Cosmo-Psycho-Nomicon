\section{エンジニアリングが持つダイナミズムからの疎外の結果1(抑圧と反抗)}\label{ux30a8ux30f3ux30b8ux30cbux30a2ux30eaux30f3ux30b0ux304cux6301ux3064ux30c0ux30a4ux30caux30dfux30baux30e0ux304bux3089ux306eux758eux5916ux306eux7d50ux679cuxff11ux6291ux5727ux3068ux53cdux6297}

\subsection{複雑化・硬直・プレモダン}\label{ux8907ux96d1ux5316ux786cux76f4ux30d7ux30ecux30e2ux30c0ux30f3}

人間が構築する自然を制御するための仕組みや制度は、より単純なものを組み合わせることでより複雑なものとなっていく(\#10.1)。この複雑化の過程は、人間社会の発展に必然的に伴うものだ。なぜなら、人間は自身が生み出した仕組みや制度を通じて相互に協働しながら生きる(=「社会生活」)からである(\#10.2)。

\begin{note}{}
  \begin{itemize}
    \tightlist
    \item{\#10.1}人間が構築した自然を制御する仕組みや制度は、より単純なものを組み合わせることでより複雑なものとなる。
    \item{\#10.2}人間は、自身が生み出した仕組みや制度を通じて相互に協働しながら生きる(=「社会生活」)。
  \end{itemize}
\end{note}

しかし、この社会の複雑化には重要な帰結が伴う。社会が複雑化すると、特に上位の仕組みや制度に変更を加えることは次第に困難になっていく(\#10.3)。これは、社会を支える基本的な考え方や枠組み(\(\textrm{S}_1\))を失墜させることができなくなるということを意味している(\#10.4)。

\begin{note}{}
  \begin{itemize}
    \tightlist
    \item{\#10.3}社会が複雑化すると、上位の仕組みや制度に変更を加えるのは容易ではなくなっていく。
    \item{\#10.4}これは、上位の$\textrm{S}_1$を失墜させることができないということと等価である。
  \end{itemize}
\end{note}

社会を支える基本的な考え方や枠組み(\(\textrm{S}_1\))を失墜させることができない状況下では、エンジニアリングの際に見られたような四つのディスクール間の柔軟な移行が阻害される。その結果、各ディスクールはそれぞれの局面に固定されやすくなる。これは、本来であれば相互に移行し合うことで問題解決を可能にしていた四つのディスクールの機能が硬直化することを意味している(\#10.5)。このような硬直化した状態によって、近代以前の社会(プレモダン)に特徴的に見られた状態を説明できる。

\begin{note}{}
  \begin{itemize}
    \tightlist
    \item{\#10.5} $\textrm{S}_1$を失墜させることができない状況下では、エンジニアリングの際とは異なり、四つのディスクールの各局面はそのディスクールのまま固定されやすくなる。
  \end{itemize}
\end{note}

\subsection{大学のディスクール}\label{ux5927ux5b66ux306eux30c7ux30a3ux30b9ux30afux30fcux30eb}

\begin{note}{}
  \begin{itemize}
    \tightlist
    \item{\#10.6} $\textrm{S}_1$を失墜させることができない状況における大学のディスクール:
      \begin{itemize}
        \tightlist
        \item 大学のディスクールは$\textrm{S}_1$の失墜を試みないため、この状況下において大学のディスクールは最も適合的なスタンスとなる。
        \item ただし、社会に適合的であることと不満(=$a$)が解消されることとは別である。
        \item 他のディスクールに移ることを十分に学ばないまま身を持ち崩して大学のディスクールの中で評価されない周縁(=$a$)に追いやられた場合、大学のディスクールにおける自己滅却的な主体(=$\frac{a}{\cancel{\textrm{S}}}\downarrow$)(=$\textrm{S}_1//\cancel{\textrm{S}}$)は破滅的な選択肢を取るかもしれない。
      \end{itemize}

$$
\uparrow\frac{\mathrm{S_2}}{\mathrm{S_1}}\genfrac{}{}{0pt}{}{\longrightarrow}{//}\frac{a}{\cancel{\textrm{S}}}\downarrow
$$
  \end{itemize}
\end{note}

あ

\subsection{ヒステリー者のディスクール}\label{ux30d2ux30b9ux30c6ux30eaux30fcux8005ux306eux30c7ux30a3ux30b9ux30afux30fcux30eb}

\begin{note}{}
  \begin{itemize}
    \tightlist
    \item{\#10.7} $\textrm{S}_1$を失墜させることができない状況におけるヒステリー者のディスクール:
      \begin{itemize}
        \tightlist
        \item ヒステリー者のディスクールは、$\textrm{S}_1$により提供される$\textrm{S}_2$が主体の不満(=$\uparrow\frac{\cancel{\textrm{S}}}{a}$)を満足させられないことを明らかにするが、
        \item それにもかかわわらず$\textrm{S}_1$を失墜させることができないため、不満を抱えたままの状態に置かれる。
        \item 不満を持っている者同士が集まることもあるが、ヒステリー者のディスクールは新たなS1を打ち立てるものでもないため、不満を持つ者の集団から秩序が生まれることもない。
      \end{itemize}

$$
\uparrow\frac{\cancel{\textrm{S}}}{a}\genfrac{}{}{0pt}{}{\longrightarrow}{//}\frac{\mathrm{S_1}}{\mathrm{S_2}}\downarrow
$$
  \end{itemize}
\end{note}

あ

\subsection{分析家のディスクール}\label{ux5206ux6790ux5bb6ux306eux30c7ux30a3ux30b9ux30afux30fcux30eb}

\begin{note}{}
  \begin{itemize}
    \tightlist
    \item{\#10.8} $\textrm{S}_1$を失墜させることができない状況における分析家のディスクール:
      \begin{itemize}
        \tightlist
        \item ・分析家のディスクールでは、うまくいかなさ(=$a$)を抱えた当人にそのうまくいかなさを解消する$\textrm{S}_1$を生み出させる(=$\frac{\cancel{\textrm{S}}}{\textrm{S}_1}\downarrow$)ことで、当人なりの新しい世界解釈を生み出す結果につながる場合がありうる。しかし、社会の$\textrm{S}_1$を失墜させることができない状況下では、社会のあり方を変えること自体は困難なままであるし、この「新しい世界解釈」は妄想への入口である場合がある(→主人のディスクール)。
      \end{itemize}

$$
\uparrow\frac{a}{\mathrm{S_2}}\genfrac{}{}{0pt}{}{\longrightarrow}{//}\frac{\cancel{\textrm{S}}}{\mathrm{S_1}}\downarrow
$$
  \end{itemize}
\end{note}

あ

\subsection{主人のディスクール}\label{ux4e3bux4ebaux306eux30c7ux30a3ux30b9ux30afux30fcux30eb}

社会を支える基本的な考え方や枠組み(\(\textrm{S}_1\))を失墜させることができないプレモダン的な状況において、主人のディスクールはもはや珍しいものとなる。それは、確立された視点(\(\textrm{S}_1\))から新たに規定される知(\(\textrm{S}_2\))が枯渇してしまっているからだ(\#10.9)。

ただし、新しい未規定の領域が眼前に現れなくても、プレモダン的な状況において主人のディスクールが生じる場合がある。それは、(先に分析家のディスクールを経て)主体が新たな視点(=\(\textrm{S}1\))に基づく世界解釈の可能性を発見した場合だ(\#10.9)。そのような場合では、その視点(\(\textrm{S}_1\))に基づいた世界の再解釈(\(\textrm{S}_1\rightarrow\textrm{S}_2\))が行われるようになる。このような再解釈が新しい社会の礎となることもありうるが、実際の社会のあり方にそぐわない妄想的な解釈としてそうした再解釈が現れることもある。なお、往々にしてそうした妄想的な解釈は、既存の常識的な知(\(\textrm{S}_2\))を退け、自身が発見した妄想的な知(\(\textrm{S}_2\))の中に主体を閉じ込めてしまう。陰謀論に「目覚めた」人が陥っているのは、典型的にはそのような状態であると考えることができる。

\begin{note}{}
  \begin{itemize}
    \tightlist
    \item{\#10.9} $\textrm{S}_1$を失墜させることができない状況における主人のディスクール:
      \begin{itemize}
        \tightlist
        \item 確立された$\textrm{S}_1$から新たに規定される$\textrm{S}_2$が枯渇してしまっているため、主人のディスクールが発生しなくなる。
        \item ただし、分析家のディスクールを経て、新たな視点(=$\textrm{S}_1$)に基づく世界解釈の可能性を発見した場合、その$\textrm{S}_1$に基づいた世界の再解釈が行われるようになることがある(それが新しい社会の礎となることもありうるが、実際の社会のあり方にそぐわない妄想的な解釈であることもある)。
      \end{itemize}

$$
\uparrow\frac{\mathrm{S_1}}{\cancel{\textrm{S}}}\genfrac{}{}{0pt}{}{\longrightarrow}{//}\frac{\mathrm{S_2}}{a}\downarrow
$$
  \end{itemize}
\end{note}

\subsection{硬直したプレモダン的労働における疎外}\label{ux786cux76f4ux3057ux305fux30d7ux30ecux30e2ux30c0ux30f3ux7684ux52b4ux50cdux306bux304aux3051ux308bux758eux5916}

フォーディズム・設計主義
人間は機械の一部として量的に扱われる(リソース(=資材)としての労働力)
精神分析は帝国主義とフォーディズムの時代の産物かも
