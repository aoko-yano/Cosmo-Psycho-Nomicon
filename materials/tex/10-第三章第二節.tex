\section{エンジニアリングが持つダイナミズムからの疎外の結果1(抑圧と反抗)}\label{ux30a8ux30f3ux30b8ux30cbux30a2ux30eaux30f3ux30b0ux304cux6301ux3064ux30c0ux30a4ux30caux30dfux30baux30e0ux304bux3089ux306eux758eux5916ux306eux7d50ux679cuxff11ux6291ux5727ux3068ux53cdux6297}

\subsection{複雑化・硬直・プレモダン}\label{ux8907ux96d1ux5316ux786cux76f4ux30d7ux30ecux30e2ux30c0ux30f3}

\begin{note}{}
  \begin{itemize}
    \tightlist
    \item{\#10.1}人間が構築した自然を制御する仕組みや制度は、より単純なものを組み合わせることでより複雑なものとなる。
    \item{\#10.2}人間は、自身が生み出した仕組みや制度を通じて相互に協働しながら生きる(=「社会生活」)。
    \item{\#10.3}社会が複雑化すると、上位の仕組みや制度に変更を加えるのは容易ではなくなっていく。
    \item{\#10.4}これは、上位の$\textrm{S}_1$を失墜させることができないということと等価である。
    \item{\#10.5} $\textrm{S}_1$を失墜させることができない状況下では、エンジニアリングの際とは異なり、四つのディスクールの各局面はそのディスクールのまま固定されやすくなる。
  \end{itemize}
\end{note}

あ

\subsection{主人のディスクール}\label{ux4e3bux4ebaux306eux30c7ux30a3ux30b9ux30afux30fcux30eb}

\begin{note}{}
  \begin{itemize}
    \tightlist
    \item{\#10.6} $\textrm{S}_1$を失墜させることができない状況における主人のディスクール:
      \begin{itemize}
        \tightlist
        \item 確立された$\textrm{S}_1$から新たに規定される$\textrm{S}_2$が枯渇してしまっているため、新しい未既定の領域が眼前に現れない限り、(通常は)主人のディスクールが発生しなくなる。
        \item ただし、分析家のディスクールを経て、新たな視点(=$\textrm{S}_1$)に基づく世界解釈の可能性を発見した場合、その$\textrm{S}_1$に基づいた世界の再解釈が行われるようになることがある(それが端的に新奇な解釈であることもあるが、実際の社会のあり方にそぐわない妄想的な解釈であることもある)。
      \end{itemize}

$$
\uparrow\frac{\mathrm{S_1}}{\mathrm{\cancel{S}}}\genfrac{}{}{0pt}{}{\longrightarrow}{//}\frac{\mathrm{S_2}}{a}\downarrow
$$
  \end{itemize}
\end{note}

\subsection{大学のディスクール}\label{ux5927ux5b66ux306eux30c7ux30a3ux30b9ux30afux30fcux30eb}

\begin{note}{}
  \begin{itemize}
    \tightlist
    \item{\#10.7} $\textrm{S}_1$を失墜させることができない状況における大学のディスクール:
      \begin{itemize}
        \tightlist
        \item 大学のディスクールは$\textrm{S}_1$の失墜を試みないため、  この状況下において大学のディスクールは最も適合的なスタンスとなる。
        \item ただし、社会に適合的であることと不満(=$a$)が解消されることとは別である。
        \item 他のディスクールに移ることを十分に学ばないまま身を持ち崩して大学のディスクールの中で評価されない周縁(=$a$)に追いやられた場合、大学のディスクールにおける自己滅却的な主体(=$\frac{a}{\cancel{\textrm{S}}}\downarrow$)(=$\textrm{S}_1//\cancel{\textrm{S}}$)は破滅的な選択肢を取るかもしれない。
      \end{itemize}

$$
\uparrow\frac{\mathrm{S_2}}{\mathrm{S_1}}\genfrac{}{}{0pt}{}{\longrightarrow}{//}\frac{a}{\mathrm{\cancel{S}}}\downarrow
$$
  \end{itemize}
\end{note}

あ

\subsection{ヒステリー者のディスクール}\label{ux30d2ux30b9ux30c6ux30eaux30fcux8005ux306eux30c7ux30a3ux30b9ux30afux30fcux30eb}

\begin{note}{}
  \begin{itemize}
    \tightlist
    \item{\#10.8} $\textrm{S}_1$を失墜させることができない状況におけるヒステリー者のディスクール:
      \begin{itemize}
        \tightlist
        \item ヒステリー者のディスクールは、$\textrm{S}_1$により提供される$\textrm{S}_2$が主体の不満(=$\uparrow\frac{\cancel{\textrm{S}}}{a}$)を満足させられないことを明らかにするが、
        \item それにもかかわわらず$\textrm{S}_1$を失墜させることができないため、不満を抱えたままの状態に置かれる。
        \item 不満を持っている者同士が集まることもあるが、ヒステリー者のディスクールは新たなS1を打ち立てるものでもないため、不満を持つ者の集団から秩序が生まれることもない。
      \end{itemize}

$$
\uparrow\frac{\mathrm{\cancel{S}}}{a}\genfrac{}{}{0pt}{}{\longrightarrow}{//}\frac{\mathrm{S_1}}{\mathrm{S_2}}\downarrow
$$
  \end{itemize}
\end{note}

あ

\subsection{分析家のディスクール}\label{ux5206ux6790ux5bb6ux306eux30c7ux30a3ux30b9ux30afux30fcux30eb}

\begin{note}{}
  \begin{itemize}
    \tightlist
    \item{\#10.9} $\textrm{S}_1$を失墜させることができない状況における分析家のディスクール:
      \begin{itemize}
        \tightlist
        \item ・分析家のディスクールでは、うまくいかなさ(=$a$)を抱えた当人にそのうまくいかなさを解消する$\textrm{S}_1$を生み出させる(=$\frac{\cancel{\textrm{S}}}{\textrm{S}_1}\downarrow$)ことで、当人なりの新しい世界解釈を生み出す結果につながる場合がありうる(社会の$\textrm{S}_1$を失墜させることができない状況下では、社会のあり方を変えること自体は困難なままである)。
      \end{itemize}

$$
\uparrow\frac{a}{\mathrm{S_2}}\genfrac{}{}{0pt}{}{\longrightarrow}{//}\frac{\mathrm{\cancel{S}}}{\mathrm{S_1}}\downarrow
$$
  \end{itemize}
\end{note}

あ

\subsection{硬直したプレモダン的労働における疎外}\label{ux786cux76f4ux3057ux305fux30d7ux30ecux30e2ux30c0ux30f3ux7684ux52b4ux50cdux306bux304aux3051ux308bux758eux5916}

フォーディズム・設計主義
人間は機械の一部として量的に扱われる(リソース(=資材)としての労働力)
精神分析は帝国主義とフォーディズムの時代の産物かも
