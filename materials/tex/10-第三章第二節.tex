\section{エンジニアリングが持つダイナミズムからの疎外の結果1(抑圧と反抗)}\label{ux30a8ux30f3ux30b8ux30cbux30a2ux30eaux30f3ux30b0ux304cux6301ux3064ux30c0ux30a4ux30caux30dfux30baux30e0ux304bux3089ux306eux758eux5916ux306eux7d50ux679cuxff11ux6291ux5727ux3068ux53cdux6297}

\subsection{複雑化・硬直・プレモダン}\label{ux8907ux96d1ux5316ux786cux76f4ux30d7ux30ecux30e2ux30c0ux30f3}

人間が構築する自然を制御するための仕組みや制度は、より単純なものを組み合わせることでより複雑なものとなっていく(\#10.1)。この複雑化の過程は、人間社会の発展に必然的に伴うものだ。なぜなら、人間は自身が生み出した仕組みや制度を通じて相互に協働しながら生きる(=「社会生活」)からである(\#10.2)。

\begin{note}{}
  \begin{itemize}
    \tightlist
    \item{\#10.1}人間が構築した自然を制御する仕組みや制度は、より単純なものを組み合わせることでより複雑なものとなる。
    \item{\#10.2}人間は、自身が生み出した仕組みや制度を通じて相互に協働しながら生きる(=「社会生活」)。
  \end{itemize}
\end{note}

しかし、この社会の複雑化には重要な帰結が伴う。社会が複雑化すると、特に上位の仕組みや制度に変更を加えることは次第に困難になっていく(\#10.3)。これは、そのレベルにおいて社会を支える基本的な考え方や枠組み(\(\textrm{S}_1\))を失墜させることが難しくなるということを意味している(\#10.4)。

\begin{note}{}
  \begin{itemize}
    \tightlist
    \item{\#10.3}社会が複雑化すると、より上位の仕組みや制度に変更を加えるのは容易ではなくなっていく。
    \item{\#10.4}これは、そのレベルにおいて$\textrm{S}_1$を失墜させることが困難だということと等価である。
  \end{itemize}
\end{note}

社会を支える基本的な考え方や枠組み(\(\textrm{S}_1\))を失墜させることが困難な状況下では、エンジニアリングの際に見られたような四つのディスクール間の柔軟な移行が阻害される。その結果、各ディスクールはそれぞれの局面に固定されやすくなる。これは、本来であれば相互に移行し合うことで問題解決を可能にしていた四つのディスクールの機能が硬直化することを意味している(\#10.5)。このような硬直化した状態によって、近代以前の社会(プレモダン)に特徴的に見られた状態を説明できる。

\begin{note}{}
  \begin{itemize}
    \tightlist
    \item{\#10.5} $\textrm{S}_1$を失墜させることが困難な状況下では、エンジニアリングの際とは異なり、四つのディスクールの各局面はそのディスクールのまま固定されやすくなる。
  \end{itemize}
\end{note}

\subsection{硬直した状況下での大学のディスクール}\label{ux786cux76f4ux3057ux305fux72b6ux6cc1ux4e0bux3067ux306eux5927ux5b66ux306eux30c7ux30a3ux30b9ux30afux30fcux30eb}

社会や組織を支える基本的な考え方や枠組み(=\(\textrm{S}_1\))を失墜させることが困難な状況において、\index{大学のディスクール}は支配的な位置を占める。\index{大学のディスクール}は本来的にそれらの考え方や枠組み(\(\textrm{S}_1\))の失墜を試みないため、この状況下において最も適合的なスタンスとなるからだ。既存の枠組みを維持しながら、その中で可能な改善や調整を行っていくという姿勢は、社会の安定性を保つ上で重要な役割を果たす。その具体的な例としては、古臭い校則で雁字搦めになった学校の中で、時代の変化に伴う様々な校内の出来事を、専ら「どの校則を敷衍させればぞの是非を判断できるか」という観点からのみ考える教員の振る舞いなどが挙げられるだろう。たしかに、それはその学校を生きていく上では適合的な態度ではある。

ただし、このような社会や組織への適合性は、必ずしも個々の主体が抱える不満(=\(a\))の解消を意味するわけではない。むしろ、既存の枠組みを維持したまま問題に対処しようとすることで、根本的な解決が先送りにされる可能性すらある。そうした解決の先送りは、不満に燻る主体(=\(\cancel{\textrm{S}}\))を生じさせることになる。

それだけならまだしも、さらに深刻な問題として、他のディスクールへの移行を十分に学ばないまま、\index{大学のディスクール}の中で評価されない周縁(=\(a\))に追いやられるケースがある。このような場合、\index{大学のディスクール}における自己滅却的な主体(=\(\frac{a}{\cancel{\textrm{S}}}\downarrow\))は、破滅的な選択肢を取る可能性がある。これは、既存の枠組みの中で自己の価値を見出せなくなった主体が、その枠組みの中で自らを否定的に位置づけてしまう(=\(\textrm{S}_1//\cancel{\textrm{S}}\))危険性を示している。先の校則に縛られた学校の場合、校則至上主義的な価値観を内面化した生徒が、自身の校則への不適合に自責の念を感じて、心身の健康を害してしまう事例などを、そうした危険性の例として挙げることができる。

\begin{note}{}
  \begin{itemize}
    \tightlist
    \item{\#10.6} $\textrm{S}_1$を失墜させることが困難な状況における大学のディスクール:
      \begin{itemize}
        \tightlist
        \item 大学のディスクールは基本的な考え方や枠組み(=$\textrm{S}_1$)の失墜を試みないため、この状況下において大学のディスクールは最も適合的なスタンスとなる。
        \item ただし、社会に適合的であることと不満(=$a$)が解消されることとは別である。
        \item 他のディスクールに移ることを十分に学ばないまま身を持ち崩して大学のディスクールの中で評価されない周縁(=$a$)に追いやられた場合、大学のディスクールにおける自己滅却的な主体(=$\frac{a}{\cancel{\textrm{S}}}\downarrow$)(=$\textrm{S}_1//\cancel{\textrm{S}}$)は破滅的な選択肢を取るかもしれない。
      \end{itemize}

$$
\uparrow\frac{\mathrm{S_2}}{\mathrm{S_1}}\genfrac{}{}{0pt}{}{\longrightarrow}{//}\frac{a}{\cancel{\textrm{S}}}\downarrow
$$
  \end{itemize}
\end{note}

\subsection{硬直した状況下でのヒステリー者のディスクール}\label{ux786cux76f4ux3057ux305fux72b6ux6cc1ux4e0bux3067ux306eux30d2ux30b9ux30c6ux30eaux30fcux8005ux306eux30c7ux30a3ux30b9ux30afux30fcux30eb}

\index{ヒステリー者のディスクール}に立つ者は「こんな世の中は、あるべき状態だとは言えない。間違っている」という感覚を抱く。それは、\index{ヒステリー者のディスクール}が、既存の枠組み(=\(\textrm{S}_1\))によって提供される解決策(=\(\textrm{S}_2\))が、主体の不満(=\(\uparrow\frac{\cancel{\textrm{S}}}{a}\))を十分に満足させられないことを明らかにするものだからだ。このディスクールは、人間の知が本質的に予測誤差から逃れられないことに依拠して成立している。

しかし、ヒステリー者が間違った世の中を成り立たせている原因あるいは根拠を失墜させることが困難な状況(=社会を支える基本的な考え方や枠組み(=\(\textrm{S}_1\))を失墜させることが困難な状況)がある。このような硬直した状況では、\index{ヒステリー者のディスクール}に基づいて表明される不満は建設的な変化につながらなくなってしまう。\index{ヒステリー者のディスクール}が持つ最もユニークで建設的な作用は、既存の考え方や枠組み(\(\textrm{S}_1\))を失墜させて、新たな考え方や枠組み(\(\textrm{S}_1\))が台頭する余地を作ることにあるからだ。そうすると、主体は不満を抱えたままの状態に置かれ続けることになる。制度や規則の問題点を指摘しても、その根本的な変更は望めず、結果として不満は慢性化していく。

ところで、\index{ヒステリー者のディスクール}は新たな枠組み(\(\textrm{S}_1\))を打ち立てるものではないため、\index{ヒステリー者のディスクール}に立つ者が集まって集団を形成しても、そこから新たな秩序が生まれることはない。むしろ、不満の共有と増幅が繰り返されるだけに終わってしまう。例えば、職場の不満をSNSで共有し合う集団が形成されても、それは単なる愚痴の言い合いに終始し、実際の職場環境などの改善にはつながらないといった状況が、その典型例として挙げられる。

\begin{note}{}
  \begin{itemize}
    \tightlist
    \item{\#10.7} $\textrm{S}_1$を失墜させることが困難な状況におけるヒステリー者のディスクール:
      \begin{itemize}
        \tightlist
        \item ヒステリー者のディスクールは、基本的な考え方や枠組み(=$\textrm{S}_1$)により提供される知や仕組みなど(=$\textrm{S}_2$)が主体の不満(=$\uparrow\frac{\cancel{\textrm{S}}}{a}$)を解消しないことを明らかにするが、
        \item それにもかかわわらず基本的な考え方や枠組み($\textrm{S}_1$)を失墜させることが困難なため、不満を抱えたままの状態に置かれる。
        \item 不満を持っている者同士が集まることもあるが、ヒステリー者のディスクールは新たなS1を打ち立てるものでもないため、不満を持つ者の集団から秩序が生まれることもない。
      \end{itemize}

$$
\uparrow\frac{\cancel{\textrm{S}}}{a}\genfrac{}{}{0pt}{}{\longrightarrow}{//}\frac{\mathrm{S_1}}{\mathrm{S_2}}\downarrow
$$
  \end{itemize}
\end{note}

\subsection{硬直した状況下での分析家のディスクール}\label{ux786cux76f4ux3057ux305fux72b6ux6cc1ux4e0bux3067ux306eux5206ux6790ux5bb6ux306eux30c7ux30a3ux30b9ux30afux30fcux30eb}

\index{分析家のディスクール}は、うまくいかなさ(=\(a\))を抱えた当人に対して、そのうまくいかなさを解消するような新たな視点(=\(\textrm{S}_1\))を生み出させる(=\(\frac{\cancel{\textrm{S}}}{\textrm{S}_1}\downarrow\))ものだ。エンジニアリングのプロセスが稼働している場合は、そうした新たな視点は社会の新しい基礎になることがあった。しかし、社会を支える基本的な考え方や枠組み(\(\textrm{S}_1\))を失墜させることが困難な状況においては、そうした展開が起こることは期待できない。

そのような状況下において、\index{分析家のディスクール}を通じて、当人なりの新しい世界解釈を生み出す結果につながる場合がある。こうした「当人なりの新しい世界解釈」の事例として、常に根拠のない不安感に苛まれ続けてきた人が、その辛さ(=\(a\))について「どういう時にその感覚は強くなるのか、その感覚はどうしたら癒されるか」といったことを具体的(=\(\textrm{S}_2\))にああでもない、こうでもないと考え続けた(=\(\cancel{\textrm{S}}\))場合について考えてみよう。そこで、その人が「自分の苦しみは幼少期に受けた虐待的な家庭環境に起因するのだ、その時の辛さが反復しているのだ」と思い至った場合、その人の人生は幼少期の家庭環境を新たな座標軸として(=\(\textrm{S}_1\))再構築されることになる。第一部の序論で引用した言葉だが「人間は欲し\bou{ない}よりは、まだしも\bou{無}を欲する」(註1)のであるから、少なくともこうして苦しみには「意味」が与えられ、その人が自分を苛む苦しさに困惑することも減るだろう。

しかし、このようにして得られる新たな視点には、危険な側面もある。個人の内面における「新しい世界解釈」は必ずしも「正しい」とは限らないため、妄想的な世界解釈への入口となってしまう危険性があるからだ(これは\index{主人のディスクール}の項で述べた妄想的解釈の問題につながっていく)。

\begin{note}{}
  \begin{itemize}
    \tightlist
    \item{\#10.8} $\textrm{S}_1$を失墜させることが困難な状況における分析家のディスクール:
      \begin{itemize}
        \tightlist
        \item ・分析家のディスクールでは、うまくいかなさ(=$a$)を抱えた当人にそのうまくいかなさを解消する$\textrm{S}_1$を生み出させる(=$\frac{\cancel{\textrm{S}}}{\textrm{S}_1}\downarrow$)ことで、当人なりの新しい世界解釈を生み出す結果につながる場合がありうる。しかし、社会の$\textrm{S}_1$を失墜させることが困難な状況下では、社会のあり方を変えること自体は困難なままであるし、この「新しい世界解釈」は妄想への入口である場合がある(→主人のディスクール)。
      \end{itemize}

$$
\uparrow\frac{a}{\mathrm{S_2}}\genfrac{}{}{0pt}{}{\longrightarrow}{//}\frac{\cancel{\textrm{S}}}{\mathrm{S_1}}\downarrow
$$
  \end{itemize}
\end{note}

\begin{itemize}
\tightlist
\item
  (註1) ニーチェ(1887=1940:271)\cite{Nietzsche2}参照。
\end{itemize}

\subsection{硬直した状況下での主人のディスクール}\label{ux786cux76f4ux3057ux305fux72b6ux6cc1ux4e0bux3067ux306eux4e3bux4ebaux306eux30c7ux30a3ux30b9ux30afux30fcux30eb}

社会を支える基本的な考え方や枠組み(\(\textrm{S}_1\))を失墜させることが困難なプレモダン的な状況において、\index{主人のディスクール}はその場の前提であり、新たに発生させることは困難になる。それは、確立された視点(\(\textrm{S}_1\))から一通りすべてのものが規定され尽くした後では、新たに規定される知(\(\textrm{S}_2\))すらも枯渇してしまっているからだ(\#10.9)。主人の名において語るべきことなどもはやなく、一見したところ目新しい出来事が起きたとしても、それは\index{大学のディスクール}にしたがって処理されることになるだろう。

ただし、そのようなプレモダン的な状況においても\index{主人のディスクール}が生じる場合がある。それは、(先に\index{分析家のディスクール}を経て)主体が新たな視点(=\(\textrm{S}_1\))に基づく世界解釈の可能性を発見した場合だ(\#10.9)。そのような場合では、その視点(\(\textrm{S}_1\))に基づいた世界の再解釈(\(\textrm{S}_1\rightarrow\textrm{S}_2\))が行われるようになる。このような再解釈は新しい社会の礎となることもありうるが、「実際の社会のあり方にそぐわない妄想的な解釈」としてそうした再解釈が現れることもある。なお、往々にしてそうした妄想的な解釈は、既存の常識的な知(\(\textrm{S}_2\))を退け、自身が発見した妄想的な知(\(\textrm{S}_2\))の中に主体を閉じ込めてしまう。陰謀論に「目覚めた」人が陥っているのは、典型的にはそのような状態であると考えることができる。

\begin{note}{}
  \begin{itemize}
    \tightlist
    \item{\#10.9} $\textrm{S}_1$を失墜させることが困難な状況における主人のディスクール:
      \begin{itemize}
        \tightlist
        \item 確立された$\textrm{S}_1$から新たに規定される$\textrm{S}_2$が枯渇してしまっているため、主人のディスクールの発生は困難になる。
        \item ただし、分析家のディスクールを経て、新たな視点(=$\textrm{S}_1$)に基づく世界解釈の可能性を発見した場合、その視点($\textrm{S}_1$)に基づいた世界の再解釈が行われるようになることがある(それが新しい社会の礎となることもありうるが、実際の社会のあり方にそぐわない妄想的な解釈に終わることもある)。
      \end{itemize}

$$
\uparrow\frac{\mathrm{S_1}}{\cancel{\textrm{S}}}\genfrac{}{}{0pt}{}{\longrightarrow}{//}\frac{\mathrm{S_2}}{a}\downarrow
$$
  \end{itemize}
\end{note}

\subsection{硬直したプレモダン的労働における疎外}\label{ux786cux76f4ux3057ux305fux30d7ux30ecux30e2ux30c0ux30f3ux7684ux52b4ux50cdux306bux304aux3051ux308bux758eux5916}

このように描かれる「\(\textrm{S}_1\)を失墜させることが困難な状況」がもたらす弊害は、まさに「フォード生産システム(註1)」的な組織に現れる弊害と同じだと言えるだろう。フォード生産システムは、標準化された生産システムを厳格に維持し、その基本的な枠組みの変更を許さない硬直的なシステムとして機能しがちだからである。そこでは、「既存の生産システムの枠内での効率化や改善のみが追求(\index{大学のディスクール})」されがちで、「労働者の不満が表明されても、システムの根本的な変更への機運は上がりにくく(\index{ヒステリー者のディスクール})」、改善活動は「個人レベルでの適応や解釈の変更に留まり、システム全体の変革には至らない(\index{分析家のディスクール})」からである。そこで「新しい仕組みを現実的に策定する(\index{主人のディスクール})」のは、その権力を持った、ある程度高い地位の役職者に限られることになる。

そして、そうした役職者は、往々にして「設計主義(註2)」的な態度を取ることになる。なぜなら、それぞれの現場に組織を変革する権限が与えられていない場合では、末端に至るまでその変革の内容を設計主義的な設計者が考えなくてはならなくなるからだ。そして、現場に裁量が与えられていないのならば、ある末端と他の末端がどのように協調するべきかも同様に設計主義的な設計者が考えなければならなくなる。そのため、設計主義的な思考は、システムを上から設計し、その枠組みを固定化する傾向が生まれる。このような設計主義がはびこり始めると、末端の側では組織を身の丈に合わせて少しずつ柔軟に変えるためのノウハウや文化が失われていく。すると、設計主義的な設計者はますます末端に裁量など与えるわけにはいかないと考え出すこともある。こうして、組織は「硬直化」していくのだ。

こうした組織でこそ、古典的な「有無を言わさず従うしかない強制的な労働」がもっとも顕著に現れる。しかし、人類に失敗についての事例が蓄積され、その克服の記憶も蓄積されるにつれて、そうした労働の支えになっていた権威は常に「このあらたな形態にあっても精神は、同様に囚われることなく(中略)そこからふたたびじぶんを大きく育てていかなければならない(註3)」と語られるようになり、\bou{資本主義}が台頭すると――すなわち、資本が\(\textrm{S}_1\)の位置に立つようになると――ついには「神は死んだ(註4)」と語られるようになる。その結果、私たちの生がどうなったのかについて、第三章第三節では見ていくことにする。

\begin{itemize}
\tightlist
\item
  (註1) 三谷(2013)\cite{Mitani}参照。
\item
  (註2)
  設計主義とその問題点、およびその乗り越え方については広木(2018)\cite{Hiroki}を参照。
\item
  (註3) ヘーゲル(1807=2018: 589)\cite{Hegel2}参照。
\item
  (註4) ニーチェ(1921=1993: 220)\cite{Nietzsche3}参照。
\end{itemize}

\subsection{この章のまとめ}\label{ux3053ux306eux7ae0ux306eux307eux3068ux3081}

本章では、エンジニアリングが持つダイナミズムが失われた状況、すなわち社会を支える基本的な考え方や枠組み(\(\textrm{S}1\))を失墜させることが困難な状況について考察した。この状況は以下のような特徴を持つ:

\begin{enumerate}
\def\labelenumi{\arabic{enumi}.}
\tightlist
\item
  社会の複雑化に伴う硬直化
\end{enumerate}

\begin{itemize}
\tightlist
\item
  人間社会の発展に伴い、制御の仕組みや制度は複雑化していく
\item
  複雑化した社会では、基本的な枠組みの変更が困難になる
\item
  その結果、四つのディスクールの相互移行が阻害され、各ディスクールは固定化する
\end{itemize}

\begin{enumerate}
\def\labelenumi{\arabic{enumi}.}
\setcounter{enumi}{1}
\tightlist
\item
  各ディスクールの硬直化した様相
\end{enumerate}

\begin{itemize}
\tightlist
\item
  \index{大学のディスクール}:既存の枠組みを維持したまま部分的な改善を行うことで、不満の根本的解決を先送りにする
\item
  \index{ヒステリー者のディスクール}:システムへの不満を表明しても建設的な変化を生み出せず、不満の共有と増幅に終始する
\item
  \index{分析家のディスクール}:個人レベルでの新たな解釈は可能でも、社会変革には至らず、時に妄想的解釈を生む
\item
  \index{主人のディスクール}:新たな規定の可能性が枯渇し、例外的に生じる場合も現実との乖離を抱えやすい
\end{itemize}

\begin{enumerate}
\def\labelenumi{\arabic{enumi}.}
\setcounter{enumi}{2}
\tightlist
\item
  フォーディズムと設計主義への帰結
\end{enumerate}

\begin{itemize}
\tightlist
\item
  硬直化した状況は、フォード生産システムに典型的に見られる
\item
  設計主義的な思考が支配的となり、現場の柔軟性が失われる
\item
  人間は「機械の一部」として量的に扱われ、労働からの疎外が進行する
\end{itemize}

このような状況は、近代以前の社会(プレモダン)に特徴的に見られた状態と類似している。しかし、人類の失敗と克服の経験の蓄積により、こうした硬直的なシステムを支える権威は次第に揺らぎ始めている。
