\section{人間の行為}\label{ux4ebaux9593ux306eux884cux70ba}

\begin{note}{}
  \begin{itemize}
    \tightlist
    \item{\#ex.1}
      対象aの顕現に対する主体による防衛としての葛藤の解消は、自我とリアリティに対する修正を伴う。
    \item{\#ex.2}
      \mbox{自由エネルギー原理}\index{じゆうえねるぎーげんり@自由エネルギー原理}における誤差の最小化と、  主体による対象aの顕現の回避のための自我とリアリティに対する修正としての葛藤の解消は、等価である。
  \end{itemize}
\end{note}

知と無知の\mbox{弁証法}\index{べんしょうほう@弁証法}以外の知性は存在しない
変分ベイズ推論以外の思考は存在しない 生は無知を孕んでいる
