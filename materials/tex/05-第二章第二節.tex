\section{\texorpdfstring{対象\(a\)と欲動の主体}{対象aと欲動の主体}}\label{ux5bfeux8c61aux3068ux6b32ux52d5ux306eux4e3bux4f53}

\subsection{予測誤差、反復強迫、欲動}\label{ux4e88ux6e2cux8aa4ux5deeux53cdux5fa9ux5f37ux8febux6b32ux52d5}

予測との誤差が大きい体験、すなわち言語化されていない体験、経験に昇華されていない体験、内的体験、トラウマ、〈物〉は、その体験自体に中毒性があるため、脳において「反復」される。これを「反復強迫」と呼ぶ。この反復強迫には「享楽」が伴う。

この反復強迫を通じて、体験を予測できるようにしようとする機制が働き、これを「欲動」と呼ぶ。体験が予測できるようになって欲動が解消されると、満足がもたらされる。

\begin{note}{}
  \begin{itemize}
    \tightlist
    \item{\#5.1}予測との誤差が大きい体験
      \begin{itemize}
        \tightlist
        \item (=言語化されていない体験)
        \item (=経験に昇華されていない体験)
        \item (=「内的体験」)
        \item (=「トラウマ」)
        \item (=「〈物〉」)
      \end{itemize}は、その体験自体に中毒性があるため、脳において「反復」される(=「反復強迫」)。
    \item{\#5.2}反復脅迫には、「享楽」が伴う。
    \item{\#5.3}反復脅迫を通じて、体験を予測できるようにしようとする機制(=「欲動」)が働く。
    \item{\#5.4}体験が予測できるようになって欲動が解消されると、満足がもたらされる。
  \end{itemize}
\end{note}

\subsection{存在論と体験}\label{ux5b58ux5728ux8ad6ux3068ux4f53ux9a13}

予測誤差を体験したとき、人は概念に収まりきらない「存在」を感じる。

\begin{note}{}
  \begin{itemize}
    \tightlist
    \item{\#5.5}予測誤差を体験したとき、概念に収まりきらない「存在」を人は感じる。
  \end{itemize}
\end{note}

\subsection{原抑圧と経験に昇華されない体験}\label{ux539fux6291ux5727ux3068ux7d4cux9a13ux306bux6607ux83efux3055ux308cux306aux3044ux4f53ux9a13}

この経験に昇華されていない体験は、ランダムで無秩序なものではなく、独自の「内包」を持つ。シニフィアンの体系に参入する際には、経験に昇華されていない体験も同時に発生することになる。これは「原抑圧」「性関係の排除」「一般化排除」「疎外」「エディプス第一の時」「前エディプス期」などと呼ばれる。

\begin{note}{}
  \begin{itemize}
    \tightlist
    \item{\#5.6}経験に昇華されていない体験は、ランダムで無秩序なものではなく、独自の「内包」を持つ。
    \item{\#5.7}シニフィアンの体系に参入する際に、経験に昇華されていない体験も同時に発生することになる
      \begin{itemize}
        \tightlist
        \item (=「原抑圧」)
        \item (=「性関係の排除」)
        \item (=「一般化排除」)
        \item (=「疎外」)
        \item (=「エディプス第一の時」)  
        \item (=「前エディプス期」)。
      \end{itemize}
  \end{itemize}
\end{note}

\subsection{\texorpdfstring{対象\(a\)の顕現と不安}{対象aの顕現と不安}}\label{ux5bfeux8c61aux306eux9855ux73feux3068ux4e0dux5b89}

体験が経験へと昇華されていない状態は、「世界と体験との間に『葛藤』がある状態」だと表現できる。原抑圧により生じる、独自の内包を持った反復強迫する体験が、「対象\(a\)」(「〈物〉の断片」)である。対象\(a\)が意識に現れること(「対象\(a\)の顕現」)は、自身が採用しているシニフィアンの体系では体験を説明しきれないことを証明してしまうため、その体験を統御できない「不安」と、その不安を解消するための「防衛機制」を呼び起こす。対象\(a\)の顕現は、「大他者の非一貫性(\(\cancel{\textrm{A}}\))」(「象徴界の穴」)を露呈させる。

\begin{note}{}
  \begin{itemize}
    \tightlist
    \item{\#5.8}体験が経験へと昇華されていない状態は、「世界と体験との間に『葛藤』がある状態」だと表現できる。
    \item{\#5.9}原抑圧により生じる、独自の内包を持った反復強迫する体験が、「対象$a$」(=「〈物〉の断片」)である。
    \item{\#5.10}対象$a$が意識に現れること(=「対象$a$の顕現」)は、「自身が採用しているシニフィアンの体系では体験を説明しきれない」ことを証明してしまうため、その体験を統御できない「不安」と、その不安を解消するための「防衛機制」を呼び起こす。
    \item{\#5.11}対象aの顕現は、「大他者の非一貫性(=$\cancel{\textrm{A}}$)」(=「象徴界の穴」)を露呈させる。
  \end{itemize}
\end{note}

\subsection{主体と葛藤の解消}\label{ux4e3bux4f53ux3068ux845bux85e4ux306eux89e3ux6d88}

主体による欲動に対する防衛は速やかに行われ、この「葛藤」を解消する行為を行うものを「主体」と呼ぶ。葛藤の解消と、主体の行為と、対象\(a\)の顕現に対する防衛とは、等価である。

\begin{note}{}
  \begin{itemize}
    \tightlist
    \item{\#5.12}主体による欲動に対する防衛は速やかに行われる。
    \item{\#5.13}「葛藤」を解消する行為を行うものを「主体」と呼ぶ。
    \item{\#5.14}葛藤の解消と、主体の行為と、対象$a$の顕現に対する防衛とは、等価である。
  \end{itemize}
\end{note}

\subsection{この章のまとめ}\label{ux3053ux306eux7ae0ux306eux307eux3068ux3081}

あ
