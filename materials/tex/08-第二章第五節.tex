\section{神経症的主体における四つのディスクールと自然についての理解}\label{ux795eux7d4cux75c7ux7684ux4e3bux4f53ux306bux304aux3051ux308bux56dbux3064ux306eux30c7ux30a3ux30b9ux30afux30fcux30ebux3068ux81eaux7136ux306bux3064ux3044ux3066ux306eux7406ux89e3}

\subsection{ディスクールの要素}\label{ux30c7ux30a3ux30b9ux30afux30fcux30ebux306eux8981ux7d20}

神経症的な欲望の主体が対象\(a\)を「(〈父〉により解決される)問題」とするとき、主体はその問題に対して「(〈父〉という)根拠」と「(問題が解決された結果である)結論」が存在すると信じている(\#8.1)。このように解釈をするとき、神経症者の思考や行動を表現する四つの要素を用いて、この章で説明する「四つのディスクール」を描くことができる(\#8.2)。この四つの要素は、

\[a \rightarrow \cancel{\textrm{S}} \rightarrow \textrm{S}_1 \rightarrow \textrm{S}_2 \rightarrow a \rightarrow ...\]

という順に並べられる。つまり、「\mbox{予測誤差}\index{よそくごさ@予測誤差}が思考としての主体を作動させ、新しいシニフィアンの秩序を形成するが、そうして形成された秩序からも\mbox{予測誤差}\index{よそくごさ@予測誤差}が必ず発生する」という思考の\mbox{弁証法}\index{べんしょうほう@弁証法}がここにもみられるのだ。

\begin{note}{}
  \begin{itemize}
    \tightlist
    \item{\#8.1}
      神経症的な欲望の主体が対象aを「(〈父〉により解決される)問題」とするとき、主体はその問題に対して「(〈父〉という)根拠」と「(問題が解決された結果である)結論」が存在すると信じている。
    \item{\#8.2}
      このように解釈をするとき、
      \begin{itemize}
          \tightlist
          \item{$a$}
          対象あるいは残余$a$。転じて、\mbox{予測誤差}\index{よそくごさ@予測誤差}としての不確実性、加えて、そこにファルスが与えられるべきとされるもの
          \item{$\cancel{\textrm{S}}$}
          欲望の主体。対象$a$の解消を試みてシニフィアンを操作する
          \item{$\textrm{S}_1$}
          象徴的父。転じて、シニフィアンを連鎖させて構築する「言説」の根拠であり、根拠の選択の仕方により規定される「問いの枠組み(プロブレマティク)」
          \item{$\textrm{S}_2$}
          象徴的父がもたらす法。転じて、言説の結論であり、問いの枠組みにおいて根拠に従属する諸命題
        \end{itemize}
    の四つの要素を用いて、神経症者の思考や行動を表現する「四つのディスクール」を描くことができる。
  \end{itemize}
\end{note}

ここで付言しておくべきことは、「問いの枠組み(プロブレマティク)」は問いだけでなく、答えをも方向づけてしまうということだ(註1)。なぜならば、答えというものは常に「問いに対する答え」だからである。このことを理解するために、西洋近代哲学における「経験論(外界についての経験を積み上げることで、世界の真理に辿り着けるとする考え方)」と「合理論(疑うことができない思考の原理原則を分析することで、世界の真理に辿り着けるとする)」という伝統的な二つの潮流を取り上げる。この両者の違いは、経験論が真理を客体の側に、合理論が真理を主体の側に位置づけている点にある。しかし、それは逆に言えば、どちらの考え方も主体/客体/真理という共通した語彙(および、そこに付随する関係性)を持っているということでもある。経験論と合理論という対立は、共通した一つの「問いの枠組み」から生じた二つの答えにすぎないのだ。経験論と合理論の例についていえば、それらは確かに様々な説明を生み出す基盤となる\(\textrm{S}_1\)ではあるのだが、それらが派生的な知\(\textrm{S}_2\)としての位置におかれるような、より根源的な枠組み\(\textrm{S}_1\)としての「主体/客体/真理」という語彙が存在しているわけだ(こうした、ある世界観が持つ階層性については第三章第二節で詳しく扱う)。

そして、このプロブレマティクが西洋近代のものであるように、「問いの枠組み」は歴史的なシニフィアンの体系の変遷に伴って形成されるものだ。実際、近代以前のスコラ学では普遍的なものをめぐる「実念論」と「唯名論」の対立が普遍論争という論争を引き起こしていたし、西洋以外では西洋近代ともスコラ学とも異なる問いの枠組みがそれに起因する学問的探求を支えていた。「問いの枠組み(=\(\textrm{S}_1\))」は、その時代、その場所、その集団ごとに採用される「世界を秩序立てて説明する座標軸の取り方」であり、その座標軸の取り方によって、世界の様々なもの(=\(\textrm{S}_2\))の説明のされ方も定まり、その限界(=\(a\))も自ずと定まるのだ。

\begin{itemize}
\tightlist
\item
  (註1) 今村(2007:355-357)\cite{Imamura1}参照。
\end{itemize}

\subsection{ディスクールの構造}\label{ux30c7ux30a3ux30b9ux30afux30fcux30ebux306eux69cbux9020}

四つのディスクールを構成する各位置には、それぞれの役割がある(\#8.3)。この四つの位置は、思考の\mbox{弁証法}\index{べんしょうほう@弁証法}において「動因が他者を新たな位置に据える際に、そのような仕方で動因が動かなければならない原因が真理にあり、動因が他者を新たな位置に据えた結果として生産物が発生してしまうという」ということを意味している。

\begin{note}{}
  \begin{itemize}
    \tightlist
    \item{\#8.3}
      四つのディスクールを構成する各位置には、下記のような役割がある。
      \begin{itemize}
          \tightlist
          \item
          主体が当初同一化しているものが真理である。
          \item
          真理には十全でないところがあり、それが動因を発生させる。
          \item
          動因は他者に働きかけ、他者は生産物を算出する。
          \item
          生産物は真理を十全にすべく生じたものだが、それは実現しない。
        \end{itemize}

$$
\uparrow\frac{\mathrm{動因}}{\mathrm{真理}}\genfrac{}{}{0pt}{}{\longrightarrow}{//}\frac{\mathrm{他者}}{生産物}\downarrow
$$
  \end{itemize}
\end{note}

ここからはエディプス・コンプレックスの流れに即して四つのディスクールをみていくが、この四つのディスクールは、主体が世界を理解するプロセスに対応しているとも言える。そこで、本章では具体例として人間が自然を理解する場合ではどのように四つのディスクールが働いているのかも見ていくことにする。

\subsection{分析家のディスクール}\label{ux5206ux6790ux5bb6ux306eux30c7ux30a3ux30b9ux30afux30fcux30eb}

分離が始まる瞬間(=エディプス第二の時)に対応するのが、右の「\mbox{分析家のディスクール}\index{ぶんせきかのでぃすくーる@分析家のディスクール}」である(\#8.4)。このようなディスクールの具体例としては、新しい世界観の提唱を挙げることができるだろう。天体運動論は、その中でも特にスッキリとした数学的解決が得られた例として参照することができる。そこでは、まず所与の感覚としての夜闇と光点の動き
\(\textrm{S}_2\) がある。その乱雑な所与の感覚は不安 \(a\)
を催し、それを理解しようとする主体 \(\cancel{\textrm{S}}\)
を作動させる。そうして、地球の周りを天体が回るという「天動説」のアイデアが\(\textrm{S}_1\)として設定される。この設定によって、天空を巡る光点の群れという混沌を秩序付けられるようになる。

\begin{note}{}
  \begin{itemize}
    \tightlist
    \item{\#8.4}
      分離が始まる瞬間(=エディプス第二の時)に対応するのが、「\mbox{分析家のディスクール}\index{ぶんせきかのでぃすくーる@分析家のディスクール}」である。
      \begin{itemize}
          \tightlist
          \item
          主体は既存のシニフィアンの体系(=$\textrm{S}_2$)に同一化している。
          \item
          シニフィアンの体系には非一貫性があり、\mbox{予測誤差}\index{よそくごさ@予測誤差}としての残余$a$が生じる。
          \item
          残余$a$は主体(=$\cancel{\textrm{S}}$)を作動させ、主体は革新的な視点(=$\textrm{S}_1$)を得る
          \item
          新たな視点は既存のシニフィアンの体系と調和せず(=$\textrm{S}_2//\textrm{S}_1$)、シニフィアンの体系を組みかえはじめる
        \end{itemize}このディスクールは不安定であり、速やかに下記の「\mbox{主人のディスクール}\index{しゅじんのでぃすくーる@主人のディスクール}」へと移行する。

$$
\uparrow\frac{a}{\mathrm{S_2}}\genfrac{}{}{0pt}{}{\longrightarrow}{//}\frac{\cancel{\textrm{S}}}{\mathrm{S_1}}\downarrow
$$
  \end{itemize}
\end{note}

\subsection{主人のディスクール}\label{ux4e3bux4ebaux306eux30c7ux30a3ux30b9ux30afux30fcux30eb}

父性隠喩を確立する段階(=「エディプス第三の時」)に対応するのが、\mbox{主人のディスクール}\index{しゅじんのでぃすくーる@主人のディスクール}である(\#8.5)。天体運動論の例でいえば、天動説というアイデアに基づいて所与の感覚を天体の動きとして秩序付ける動きが始まることで、主体が抱える不安は解消に向かうことになる(=\(\uparrow\frac{\textrm{S}_1}{\cancel{\textrm{S}}}\))。そうして、そのアイデアに沿って、個々の天体がどれくらいの周期で天球上を巡るかという具体的な説明がされるようになる(=\(\textrm{S}_1\rightarrow\textrm{S}_2\))。だが、天体の軌道は地球を中心とした単純な円によってはうまく説明しきれない。惑星は天球上を逆行することもあるからだ。こうして、理論との誤差(=\(a\))が生まれる(=\(\frac{\textrm{S}_2}{a}\downarrow\))。この誤差は、素朴な天動説によっては解消することができない(\(\cancel{\textrm{S}}//a\))。

\begin{note}{}
  \begin{itemize}
    \tightlist
    \item{\#8.5}
      父性隠喩を確立する段階(=「エディプス第三の時」)に対応するのが、\mbox{主人のディスクール}\index{しゅじんのでぃすくーる@主人のディスクール}である。
      \begin{itemize}
          \tightlist
          \item
          主体(=$\cancel{\textrm{S}}$)は新たな根拠となるシニフィアン(=$\textrm{S}_1$)に依拠する。
          \item
          新たな根拠に基づいて様々な命題が生み出されていく(=$\textrm{S}_1\rightarrow\textrm{S}_2$)。
          \item
          しかし、そうして構築された新たなシニフィアンの体系にも非一貫性(=$a$)がある。
          \item
          この非一貫性は、このディスクールで最初に欲望の主体が解消しようとしたものとは異なる新たな残余$a$である。
          \item
          生み出された残余$a$と主体との間には断絶があるが(=$\cancel{\textrm{S}}//a$)、主体はこの断絶が克服されうるものなのだという幻想を信じている(=$\cancel{\textrm{S}}$◇a)。
        \end{itemize}

$$
\uparrow\frac{\mathrm{S_1}}{\cancel{\textrm{S}}}\genfrac{}{}{0pt}{}{\longrightarrow}{//}\frac{\mathrm{S_2}}{a}\downarrow
$$
  \end{itemize}
\end{note}

惑星の動きほどスッキリとした解決が得られる場合は少ないが、自然の動きについても本章でみるのと同じプロセスを経て理解が進むことになる。まずは混沌とした所与の中に、秩序の源泉となる点が打ち立てられる(=\mbox{分析家のディスクール}\index{ぶんせきかのでぃすくーる@分析家のディスクール})。そして、そこを根拠として物事が体系的に説明されるようになり、それに合わせて人は制度や仕組みを作るようになるのだ(=\mbox{主人のディスクール}\index{しゅじんのでぃすくーる@主人のディスクール})。

なお、ある対象の範囲について物事を説明するアイデア(=\(\textrm{S}_1\))が打ち立てられるとき、そのアイデアはより根源的に社会を支える別のアイデア(=\(\textrm{S}_1\))との整合性も問われることになる(繰り返しになるが、こうした世界観の階層性については第三章第二節で詳しく扱う)。近代以前の西洋において、天動説というアイデア(=\(\textrm{S}_1\))は、より社会を根源的に支えていたキリスト教的世界観(=\(\textrm{S}_1\))とも整合的なものだった。たとえば、地球を宇宙の中心に据える世界解釈は、「神の天地創造により生まれた特別な存在としての地」という地球についての解釈と整合的であった。また、旧約聖書内の記述(『ヨシュア記』10章13節など)も地球が不動の天体である根拠とされた(註1)。

\begin{itemize}
\tightlist
\item
  (註1) 船越(1885)\cite{Funakoshi}参照。
\end{itemize}

\subsection{大学のディスクール}\label{ux5927ux5b66ux306eux30c7ux30a3ux30b9ux30afux30fcux30eb}

確立した父性隠喩について、現実的父に同一化し象徴的ファルスを持っていると思いたい者は「\mbox{大学のディスクール}\index{だいがくのでぃすくーる@大学のディスクール}」を好むようになる(\#8.6)。\mbox{大学のディスクール}\index{だいがくのでぃすくーる@大学のディスクール}は、\mbox{分析家のディスクール}\index{ぶんせきかのでぃすくーる@分析家のディスクール}や\mbox{主人のディスクール}\index{しゅじんのでぃすくーる@主人のディスクール}とは異なり、新たな枠組みの創始を望まない官僚主義的なディスクールだ(註1)。例えば、天動説という世界観(=\(\textrm{S}_1\))に基づいた説明(=\(\uparrow\frac{\textrm{S}_2}{\textrm{S}_1}\))では、説明できないままに留まっていた惑星の軌道(=\(a\))に対して、歴史上いくつかの概念が追加された。そうすることによって、理論は精緻化され、\mbox{予測誤差}\index{よそくごさ@予測誤差}を縮小することができたからだ(註2)。実際、それらの概念を追加することによって惑星の逆行もかなり精度よく計算できるようになる。だが、\mbox{予測誤差}\index{よそくごさ@予測誤差}がゼロになることはなく、したがって認識を発展させる\mbox{弁証法}\index{べんしょうほう@弁証法}的作用としての主体(=\(\cancel{\textrm{S}}\))も燻り続ける(=\(\frac{a}{\cancel{\textrm{S}}}\downarrow\))。とはいえ、追加された概念によって天体の軌道予測は精緻にし続けていくことができるのだから、\mbox{大学のディスクール}\index{だいがくのでぃすくーる@大学のディスクール}に立つ主体は、あえて天動説を放棄して混沌を再び解き放とうとは思わないだろう(=\(\textrm{S}_1//\cancel{\textrm{S}}\))。

\begin{note}{}
  \begin{itemize}
    \tightlist
    \item{\#8.6}
      確立した父性隠喩について、現実的父に同一化し象徴的ファルスを持っていると思いたい者は「\mbox{大学のディスクール}\index{だいがくのでぃすくーる@大学のディスクール}」を好むようになる。
      \begin{itemize}
          \tightlist
          \item
          主体(=$\cancel{\textrm{S}}$)は言説の根拠(=$\textrm{S}_1$)を所持する者に同一化している。
          \item
          言説の根拠はそれ単独ではシニフィアンの体系を形成できず、自身に基づいた様々な命題を持っている(=$\uparrow\frac{\textrm{S}_2}{\textrm{S}_1}$)。
          \item
          様々な命題は、新たな残余$a$を既存の問いの枠組みを保持したまま解決しようとする(=$\textrm{S}_2\rightarrow a$)。
          \item
          だが、その試みは不徹底に終わり、新たな欲望の主体(=$\cancel{\textrm{S}}$)を発生させる。
          \item
          しかし、新たな欲望の主体に従って再びシニフィアンの体系を組みかえることは、現在の主体の同一化を放棄させることを意味するので、この新たな欲望の主体は抑圧される。
        \end{itemize}

$$
\uparrow\frac{\mathrm{S_2}}{\mathrm{S_1}}\genfrac{}{}{0pt}{}{\longrightarrow}{//}\frac{a}{\cancel{\textrm{S}}}\downarrow
$$
  \end{itemize}
\end{note}

\begin{itemize}
\tightlist
\item
  (註1) 向井(2016: 381)\cite{Mukai}を参照。
\item
  (註2)
  精緻化された天動説ではは「離心円(地球の中心とは違う点に中心を持つ円)」と「従円(周転円に比べて大きな円。この従円が離心円になっている)・周転円(従円の円周上に中心を持つ点)・エカント(地球の中心とも従円の中心とも違う場所に、エカント点を打ち、周転円の中心をエカントから見て一定の角速度で動くようにする)」の概念が必要となる。詳しい説明はホイル(1973=1974)\cite{Hoyle}を参照。今日の私たちは、惑星は太陽を焦点とする楕円軌道を回るということ、その際に惑星は面積速度が一定になるように運動することなどを知っている。そのことを知っているからこそ、そこから計算して天体の軌道を計算することなど簡単だと思い込んでいる。だが、地上から見える光点の不可解な動きのみを出発点としてこの科学的知見に至ることは大変難しい。
\end{itemize}

\subsection{ヒステリー者のディスクール}\label{ux30d2ux30b9ux30c6ux30eaux30fcux8005ux306eux30c7ux30a3ux30b9ux30afux30fcux30eb}

確立した父性隠喩について、象徴的ファルスに同一化し現実的父に欲望されることを欲望する者は、「\mbox{ヒステリー者のディスクール}\index{ひすてりーしゃのでぃすくーる@ヒステリー者のディスクール}」を好むようになる(\#8.7)。\(\textrm{S}_1\)は超越的な対象についてのシニフィアンであるが、それ自体が超越的であるわけではないため、その権威を失墜させることができるという点がポイントだ。

\begin{note}{}
  \begin{itemize}
    \tightlist
    \item{\#8.7}
      確立した父性隠喩について、象徴的ファルスに同一化し現実的父に欲望されることを欲望する者は右の「\mbox{ヒステリー者のディスクール}\index{ひすてりーしゃのでぃすくーる@ヒステリー者のディスクール}」を好むようになる。
      \begin{itemize}
          \tightlist
          \item
          主体は、対象$a$の位置に来るべき象徴的ファルスに同一化するために、ファルスに仮装する(=$\uparrow\frac{\cancel{\textrm{S}}}{a}$)。
          \item
          仮装した主体は自身では対象$a$を解消できない。
          \item
          仮装した主体は対象$a$を解消すべく、現実的父になりえそうな他者に働きかけて(=$\cancel{\textrm{S}}\rightarrow\textrm{S}_1$)様々な命題を吐き出させる(=$\frac{\textrm{S}_1}{\textrm{S}_2}\downarrow$)。
          \item
          しかし、いかなる命題も対象$a$そのものを根絶することはない(=$a//\textrm{S}_2$)。
          \item
          そのため、それらの命題の根拠(=$\textrm{S}_1$)も失墜する。
        \end{itemize}

$$
\uparrow\frac{\cancel{\textrm{S}}}{a}\genfrac{}{}{0pt}{}{\longrightarrow}{//}\frac{\mathrm{S_1}}{\mathrm{S_2}}\downarrow
$$
  \end{itemize}
\end{note}

このような\mbox{ヒステリー者のディスクール}\index{ひすてりーしゃのでぃすくーる@ヒステリー者のディスクール}は、世界を理解するプロセスにおいては「\mbox{予測誤差}\index{よそくごさ@予測誤差}を常に前面に掲げ続け、既存の世界観の限界を明らかにし、その正当性や信頼を失墜させる」という革新的な動きをもたらす。そこでは、まず\mbox{予測誤差}\index{よそくごさ@予測誤差}あるいは不確実性(=\(a\))の解決(=\(\uparrow\frac{\cancel{\textrm{S}}}{a}\))を、既に確立された視点/問題の枠組み/権威(=\(\textrm{S}_1\))によって達成しようとする。だが、\(\textrm{S}_1\)は有限の知(=\(\textrm{S}_2\))しか生みだせず(=\(\frac{\textrm{S}_1}{\textrm{S}_2}\downarrow\))、それが\mbox{予測誤差}\index{よそくごさ@予測誤差}や不確実性を解決することはない(=\(a//\textrm{S}_2\))。こうして、\mbox{ヒステリー者のディスクール}\index{ひすてりーしゃのでぃすくーる@ヒステリー者のディスクール}は既存の世界観を頼ることで、かえって逆にその無能力を露呈させてしまう。その結果は\(\textrm{S}_1\)に対する失望に終わり、\(\textrm{S}_1\)は手段としての信頼を失墜させる。この\(\textrm{S}_1\)の失墜の結果、ヒステリー者は混乱の中に投げ出されてしまう。

周転円をいくら追加しても\mbox{予測誤差}\index{よそくごさ@予測誤差}がゼロにできない中でも、背後ではブラーエ(1546-1601)らによってそれまでよりも精密な計測データ
\(\textrm{S}_2\)
が溜まっていっていた。これらの計測データもまた、既存の世界観の中で用いられていた既存の計測機器を使用して貯められてきたきたものだ(=\(\frac{\textrm{S}_1}{\textrm{S}_2}\downarrow\))。こうして貯められたデータが、新しい天体モデルの誕生を準備することになった。

\subsection{分析家のディスクール、再び}\label{ux5206ux6790ux5bb6ux306eux30c7ux30a3ux30b9ux30afux30fcux30ebux518dux3073}

このような蓄積された知と混乱の中で、\mbox{分析家のディスクール}\index{ぶんせきかのでぃすくーる@分析家のディスクール}を通じて、人は新しい問題の枠組みを生み出す。天動説の例でいえば、天動説を唱えていたプトレマイオス(83頃-163頃)の著作についての批判的研究からコペルニクス(1473-1543)が太陽中心説(=地動説)を唱えた。しかし、この時点の地動説ではエカントが除去されたのみで、周転円の数が減ったわけでもなく、精度面でもそれまでの天動説より優位に立てたわけではなかった。つまり、コペルニクスの時点では地動説は完成していなかったのだ。地動説の完成は、ブラーエのデータをもとに天体の軌道を計算し、その軌道上で面積速度一定の法則などを発見したケプラー(1571-1630)の登場や、慣性の法則や金星の満ち欠けを発見したガリレイ(1564-1642)の登場、そして微積分法・古典力学・万有引力の法則を発見したニュートン(1642-1727)らの登場を待たねばならない。ニュートン力学というより根源的な枠組み\(\textrm{S}_1\)の成立によって、物体の運動が万有引力と運動方程式によって基礎付けられることとなり、そこから楕円軌道が説明されて、地動説が遂にその優位を確立するからである。

ただし、それらの発見によって地動説が完成し、天体の運動がより簡潔に説明できるようになったとしても、それは天動説が「論破」されたということを意味するわけではない。世界に起こる現象を説明するやり方は一通りではないからだ。例えば、「すべての物事は完全に自分中心で動いているのであって、そうでなうように思う他人はすべて錯覚に陥っているのだ。単純なモデルでは表せないことかもしれないが、実はそうなのだ」と矛盾なく言い張ることはできる。ただ、そうした不必要に複雑なモデルは、有用さの観点で劣っているため、世界を説明する手段として採用する者がいなくなりがちだというだけだ。こうして、人々が採用する世界観において、支持者の喪失と獲得を通じた「パラダイム転換(註1)」が起こることになる。

\begin{itemize}
\tightlist
\item
  (註1) クーン(1970=1971)\cite{Khun}を参照。
\end{itemize}

\subsection{この章のまとめ}\label{ux3053ux306eux7ae0ux306eux307eux3068ux3081}

本章では、神経症的主体における四つのディスクールについて、自然科学(特に天体運動論)を例に取りながら説明した。四つのディスクールは、主体が世界を理解していく過程で以下のように循環的に現れる:

\begin{enumerate}
\def\labelenumi{\arabic{enumi}.}
\tightlist
\item
  \mbox{分析家のディスクール}\index{ぶんせきかのでぃすくーる@分析家のディスクール}:

  \begin{itemize}
  \tightlist
  \item
    既存の理解(\(\textrm{S}_2\))に起因して\mbox{予測誤差}\index{よそくごさ@予測誤差}(\(a\))が生じることで、主体(\(\cancel{\textrm{S}}\))が作動し、新たな視点(\(\textrm{S}_1\))を見出す段階
  \item
    例:天体の動きという混沌から天動説という新しい世界観の誕生
  \end{itemize}
\item
  \mbox{主人のディスクール}\index{しゅじんのでぃすくーる@主人のディスクール}:

  \begin{itemize}
  \tightlist
  \item
    新たな視点(\(\textrm{S}_1\))に基づいて体系的な説明(\(\textrm{S}_2\))を構築する段階
  \item
    しかし、新たな\mbox{予測誤差}\index{よそくごさ@予測誤差}(\(a\))も生まれる
  \item
    例:天動説に基づく天体運動の体系的説明の確立
  \end{itemize}
\item
  \mbox{大学のディスクール}\index{だいがくのでぃすくーる@大学のディスクール}:

  \begin{itemize}
  \tightlist
  \item
    既存の枠組み(\(\textrm{S}_1\))を維持したまま、その中で説明(\(\textrm{S}_2\))を精緻化していく段階
  \item
    \mbox{予測誤差}\index{よそくごさ@予測誤差}(\(a\))は完全には解消されないが、枠組みは保持される
  \item
    例:周転円などの概念追加による天動説の精緻化
  \end{itemize}
\item
  \mbox{ヒステリー者のディスクール}\index{ひすてりーしゃのでぃすくーる@ヒステリー者のディスクール}:

  \begin{itemize}
  \tightlist
  \item
    \mbox{予測誤差}\index{よそくごさ@予測誤差}(\(a\))を前面に掲げ、既存の権威(\(\textrm{S}_1\))の限界を明らかにする段階
  \item
    既存の説明体系(\(\textrm{S}_2\))の不十分さが露呈する
  \item
    例:精密な観測データの蓄積による天動説の限界の顕在化
  \end{itemize}
\end{enumerate}

この循環を経て、再び\mbox{分析家のディスクール}\index{ぶんせきかのでぃすくーる@分析家のディスクール}が現れ、新たなパラダイムが生まれる(例:地動説の確立)。ただし、これは旧来の説明の「論破」を意味するわけではなく、より簡潔で有用な説明の採用という形でパラダイム転換が起こるのである。
