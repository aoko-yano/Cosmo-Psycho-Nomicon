\section{エンジニアリングが持つダイナミズムからの疎外の結果2(不安と暴力)}\label{ux30a8ux30f3ux30b8ux30cbux30a2ux30eaux30f3ux30b0ux304cux6301ux3064ux30c0ux30a4ux30caux30dfux30baux30e0ux304bux3089ux306eux758eux5916ux306eux7d50ux679cuxff12ux4e0dux5b89ux3068ux66b4ux529b}

\subsection{〈父の名〉の衰退}\label{ux7236ux306eux540dux306eux8870ux9000}

前節では、硬直化したシステムとそれを支える権威について見た。しかし、その権威が揺らぎ、社会における基本的な考え方や枠組み(\(\textrm{S}_1\))が効力を失っていくことがある(\#11.1)。この現象は、哲学の文脈ででは「神の死」として、ラカン派精神分析の文脈では「〈父の名〉の衰退」あるいは「象徴界の機能不全」として語られてきた状況だ。

\begin{note}{}
  \begin{itemize}
    \tightlist
    \item{\#11.1}社会において上位の$\textrm{S}_1$が効力を失う場合がある
      \begin{itemize}
        \tightlist
        \item (=「神の死」)
        \item (=「〈父の名〉の衰退」)
        \item (=「象徴界の機能不全」)
      \end{itemize}
  \end{itemize}
\end{note}

人々の行動や思考を方向づけ、社会を秩序づけてきた根拠(\(\textrm{S}_1\))が効力を失うことによって、権威の衰退以外にも様々な現象が起こる。例えば、かつての宗教的な価値観や伝統的な社会規範が持っていた拘束力が弱まり、それらが提供してきた意味や方向性が不確かなものとなっていく過程として理解できる。このような〈父の名〉の衰退は、一方では人々を既存の束縛から解放する可能性を持つが、他方では新たな問題を生み出す(註1)そうした問題について、これから見ていこう。

\begin{itemize}
\tightlist
\item
  (註1) 松本(2018b: 46-55)\cite{Matsumoto2}を参照。
\end{itemize}

\subsection{貨幣による秩序}\label{ux8ca8ux5e63ux306bux3088ux308bux79e9ux5e8f}

〈父の名〉が衰退した現代の資本主義社会において、社会を秩序付ける基本的な考え方や枠組み(\(\textrm{S}_1\))として機能しているのは「資本」である(\#11.2)(註1)。しかし、資本自体に本質的な力があるわけではない。資本が社会を秩序付ける力を持つのは、それが「労働(と、その成果)」と交換可能だからである(\#11.3)。人は、他者の\ruby{労働}{サービス}や、あるいは他者の\ruby{労働の成果}{プロダクト}を「商品」として購入するための手段として、貨幣(という代表的な資本)に価値を見出しているのだ。

\begin{note}{}
  \begin{itemize}
    \tightlist
    \item{\#11.2}資本主義社会において、社会を秩序付けるのは「資本(その代表格が貨幣)」($\textrm{S}_1$)だ。
    \item{\#11.3}資本が社会を秩序付ける力を持つのは、資本が「労働(と、その成果)」($\textrm{S}_2$)としての「商品」と交換されうるからだ。資本自体に力があるわけではない。
  \end{itemize}
\end{note}

ここでは、社会の問題や不満(=\(a\))を解決する(=\(\cancel{\textrm{S}}\))と市場において銘打たれたものであれば何でも商品となりうる(\#11.4)。つまり、貨幣は商品を介して人々の不満や欲望を媒介する役割を果たしているのである。

\begin{note}{}
  \begin{itemize}
    \tightlist
    \item{\#11.4}労働は、社会の問題(=$a$)を解決する(=$\cancel{\textrm{S}}$)と銘打たれて市場に登場する。
  \end{itemize}
\end{note}

ところで、資本主義社会において、人々は所有する資本を増大させる方向へと秩序付けられていく。そこで働いている力は、大別して二つある。まず一つは単に「より多くの資本を持つことが、社会的な権力の増大と選択可能な行動の増大に結びついている」という事情から生じる。貨幣を多く持っていればより多くの人間を動員できるし、多くの土地・材料・機械を所持していればそれだけ意図した通りの行動を取れるようになる。現時点で具体的にどのような権力や行動を行うか決めていなくても、資本を所持することは将来取りうる権力や行動の可能性を担保することになる。もう一つの力は「人類は学習することによって過去においては偉業だった行為を陳腐化させるため、同じ労働を繰り返しているだけでは社会の中で相対的に貧しくなる」という事情から生じる。絶えず効率化し続ける社会の中で落伍しないためには、絶えず自身の資本(それは、古典的なマルクス主義で注目を集めていたような工場の設備などだけではない。自身が持つ知識や健康状態もまた、それがなければ労働を危うくするという点で「資本」だと言えるのだ)を強化することを通じて、市場において過去より高い価値を他者に認めさせなければならない。

これら二つの力によって、人々は資本の増大へと方向づけられる。この過程で、人々は資本(ここでの代表例は、貨幣や時間)を「生産手段」と「生産力」の購入に再投下している。これが「資本の蓄積」と呼ばれるプロセスである。このプロセスを通じて、より高価な商品を作り出せるようになる(\#11.5)。経済的には、社会はこのような仕方でまとめられることになる(図11.2)。

\begin{note}{}
  \begin{itemize}
    \tightlist
    \item{\#11.5}資本主義社会において、人々は所有する資本を増大させる方向に秩序付けられ、資本を量的に増大させるために人々は資本を「生産手段」と「生産力」の増強に再投下し(=「資本の蓄積」)、さらに高価な商品を作ろうとする。
  \end{itemize}
\end{note}

\begin{itemize}
\tightlist
\item
  (註1) 上尾(2017:
  249-250)\cite{Ueo}を参照。上尾は資本主義のディスクールにおける\(\textrm{S}_1\)は「資本家」であるとしているが、本稿では「誰もが自身の持つ資本(例えば、自身の身体)」をマネージしているという点に着目し、\(\textrm{S}_1\)は「(すべての人間が持つことを期待されているものとしての)資本」であると捉えた。
\end{itemize}

\subsection{不安・排除・レイシズム}\label{ux4e0dux5b89ux6392ux9664ux30ecux30a4ux30b7ux30baux30e0}

このようにして作られる経済的なまとまりは、固定的な「これをしなさい」という承認の条件を与えない。こうして「ひとりひとりが自分の監督の役割を引き受ける(中略)周縁部のない」宇宙の中に人は放り出される(註1)。そこでは「父としての理想を体現する自我理想(=〈父の名〉)ではなく、母性的な超自我」が「主体の殲滅に至るまで命令や禁止を繰り返すだけであり、どこかの時点で主体に承認を与えることがない」のだ。これは父なる「『然り
oui』と告げる包摂的秩序」が失われ、母なる「『否
non』\bou{だけ}を告げる排除的秩序」が支配するようになった、ということである(註2)。神が死んでいては、「人生の詰み」は単に\bou{それだけのもの}にしかならず、そこに救済などを読み取ることも不可能になってしまう。

このような状況において、人々は新たな不安に直面することになる。それは、自分とは異なる行動パターンを取る他者(=\(a\))が、自分に危害を与えないよう社会的に統御されているという確信が揺らぐことから生じる不安(=\(\cancel{\textrm{A}}\))だ(\#11.6)。そこでは、民族的・性的など様々なマイノリティが他者として槍玉に挙げられる傾向が指摘されている。これらの他者は、「そうではない私たち」とは異なる生活様式を持っている。冠婚葬祭のやり方も違うし、日常生活でのマナーなども違う。それはつまり、これらの他者は「そうではない私たち」とは異なる仕方で感情を処理しているということでもある。このことを、他者が異なる「享楽のモード」を持つ、と言い表す(註3)。

\begin{note}{}
  \begin{itemize}
    \tightlist
    \item{\#11.6}人は上位の$\textrm{S}_1$が衰退すると、自身の理解を超えた行動パターンを取る異質な他者の行動(=$a$)が、自身に危害を与えずに社会的に統御されるとは信じられなくなり、不安(=$\cancel{\textrm{A}}$)になる。
  \end{itemize}
\end{note}

異なる享楽のモードを持つ他者は、一先ずは「社会を変えた原因」であると見なされる。そこで、「変わる前の社会」を理想化した場合、他者は「社会を本来あるべき姿から逸脱させた者」として見えてくる。そして、ここに「『(理想化された)変わる前の社会』においては、私たちは満足して暮らせていたのだ」という観念が合流すると、他者は「自分たちが本来得られたはずの享楽を妨害した上で、まんまと自分たちだけはしっかりと享楽を味わっている者ども」として現れてくる。こうして、「享楽の盗み
vol de la
jouissance」の論理が発生する(註4)。この論理に基づいて、古典的な「ヒトラーのような指導者に〈父〉のような一貫性を見出し、そこに同一化し、従う」ような「レイシズム1.0」とは異なる、現代的な「一貫性のある指導者を欠いた、異なる享楽のモードを前にしたパニックのもとで組織化されただけの、排除の運動」といえる「レイシズム2.0」が組織されることになる(註5)。異質な他者に対する「社会が本来あるべき状態になることを妨害している者」という理解が反転して生まれた「異質な他者を排除すれば社会の本来的な状態を回復させることができる」という幻想が「レイシズム」なのだ(\#11.7)。実際、レイシズム2.0の兆候が多分に認められるトランプ現象において、人々はトランプの発現内容の一貫性にではなく、彼が彼の支持者の持つ「怒り」や「恐怖」を代弁しているように見える点を評価しているのだと言われている(註6)。

\begin{note}{}
  \begin{itemize}
    \tightlist
    \item{\#11.7}この不安において、異質な他者は「社会が本来的な状態になることを妨害している者」として理解されるようになり、その理解から逆転して「異質な他者を排除すれば社会の本来的な状態を回復させることができる」という幻想が生じる(=「レイシズム」)。
  \end{itemize}
\end{note}

このレイシズム的な幻想が人々を強く惹きつける理由は、それが救いのない現実を隠蔽するからである。それは「『問題がある』と信じられるうちは『解決がある』と思える」というだけにすぎないのだが、強い満足感を与えることができる(\#11.8)。もちろん、これは単なる幻想に過ぎないのだが。このように、〈父の名〉の衰退は、経済的な秩序による置き換えだけでなく、レイシズムという新たな問題を引き起こす(図11.3)。

\begin{note}{}
  \begin{itemize}
    \tightlist
    \item{\#11.8}この幻想は、あたかも安寧な社会が実が可能であるかのように感じさせるものであるため、主体に強い満足感を与えることができる。
  \end{itemize}
\end{note}

\begin{itemize}
\tightlist
\item
  (註1) 浅田(1983: 212-3)\cite{Asada}参照。
\item
  (註2) 松本(2018a: 253)\cite{Matsumoto1}参照。
\item
  (註3) 松本(2018a: 233-4)\cite{Matsumoto1}参照。
\item
  (註4) 松本(2018a: 234-7)\cite{Matsumoto1}参照。
\item
  (註5) 松本(2018a: 219-32)\cite{Matsumoto1}参照。
\item
  (註6) Zito(2016)\cite{Zito}参照。
\end{itemize}

\subsection{資本主義のディスクール}\label{ux8cc7ux672cux4e3bux7fa9ux306eux30c7ux30a3ux30b9ux30afux30fcux30eb}

資本主義社会における人々の行動は「資本主義のディスクール」によって書き表わすことができる。このディスクールにおいて、人々は複数の役割を同時に担うことになる。まず、先に述べたようにあ、人々は「資本(\(\textrm{S}_1\))」を増やす「資本家」としての役割を持つ。資本は、攻撃されることのない隠れた「真理」の位置にあり、何の具体的な指示も出さないままで社会の成員を支配している。そして、資本を増大させるために人々は「労働(\(\textrm{S}_2\))」を行う「労働者」として市場に参加することになる。そこで人々は、自身の労働やその成果を商品として高く他者に売りつけようとする。資本主義社会において、労働は万人が強いられているものとして「他者」の位置にある。このような情景がある一方で、この市場で商品を売る当人が、まさに他方では「不満(\(a\))」を抱えた者として市場に現れ、「消費者(=\(\cancel{\textrm{S}}\))」として商品を購入する役割も担う。不満は、商品の集まりである市場が未だに満たすことのできていない空白の部分に生じる。新しい商品が世に出ることで、かえって「まだあれができない、まだこれができない」という仕方で意識にのぼってくるのが資本主義社会に置ける不満なのだ。その意味で、不満は「成果物」の位置にある。そして、消費者による購入という行為が資本の移動を引き起こす。消費者は「動因」の位置にあると言えるだろう(\#11.9)。

\begin{note}{}
  \begin{itemize}
    \tightlist
    \item{\#11.9}資本主義における人々の行動は、「資本主義のディスクール(図11.4)」によって記述できる。ここで人々は「資本家($\textrm{S}_1$)」として資本の増大を図り、資本の命令として「労働者($\textrm{S}_2$)」になる一方で、自身が抱える「不満(=$a$)」を、商品を購入する「消費者(=$\cancel{\textrm{S}}$)」として解消する。
  \end{itemize}
\end{note}

このような資本主義社会の市場には、目まぐるしく新商品が出てきて、新たな不満が解消されてはいくが、決して世界観が変わることはない。見た目上は新しいものが次々と生まれて世界を変えていくが、資本主義のディスクール自体は全く微動だにしない。不満がない状態が夢想されたまま、そこにひたすら近づいていくように見えながらも、永遠に「まだあれができない、まだこれができない」と苦しみ続けるのだ。そしてそこに、また新商品が現れて、人々を魅惑し、購入へと走らせる\ldots\ldots(註1)。

ところで、手持ちの資本は、労働の結果として手に入れたものなのだから、使わなければもったいない。その上、早めに使えば早く使う分だけ、早いうちから長時間に渡って効力を得ることができる。資本の効果を最大化したければ、資本を手元で死蔵させてはならないのだ。このような理由で、資本主義社会に生きる人々には「資本で早く何かの商品を購入せよ」というプレッシャーがかけられている。その結果、少なくない人々が「自分が本当に欲しいものなのは何か?」という点について熟慮することなく、商品を販売する側の言い分を鵜呑みにして、商品をまんまと購入させられてしまうのだ。そして、それはそれで大抵の場合、退屈しのぎにはなる。ここで失われているのは、「自分が本当に欲しいものについての洞察」と「本来ならば他のものに使えたかもしれない資本」と「人生の残り時間」である。効力と時間を惜しんで、自己理解と資本と時間を失うのだ。資本主義のディスクールの中で、自身に固有の特異性について理解を深めていくのは容易ではないわけだ(\#11.10)。

\begin{note}{}
  \begin{itemize}
    \tightlist
    \item{\#11.10}資本主義において、消費者は外部から不満を解決する手段を提供されるため、自身に固有の特異性について思考する契機を奪われることになる。
  \end{itemize}
\end{note}

\begin{itemize}
\tightlist
\item
  (註1) 松本(2018a: 49)\cite{Matsumoto1}・松本(2018b:
  67)\cite{Matsumoto2}・上尾(2017: 249-250)\cite{Ueo}を参照。
\end{itemize}

\subsection{柔軟性のあるモダンな労働における疎外}\label{ux67d4ux8edfux6027ux306eux3042ux308bux30e2ux30c0ux30f3ux306aux52b4ux50cdux306bux304aux3051ux308bux758eux5916}

モダンな労働における疎外は、欲動の外から与えられた欲望にすり替えられることで起こる
プレモダンな労働と同じく、モダンな労働でも人は生産・消費の両面で量的なリソース(資材)として扱われる

コモンが搾取や疎外を避けるわけではない。株主からかかる利潤第一主義への圧力に従属しないことがコモンズの利点だと斉藤は主張するが、それは株式会社でも可能だ。アマゾンを見ろ。株主を説得できるだけのビジョンがない知的怠惰がダメなのであって、株式会社がダメなわけではない。コモンの中でも疎外と搾取は発生しうる。改善のプロセスから逃げてはならない。

資本主義をマルクス・ガブリエルが言う「倫理資本主義」に近いものに変えると考えられる。

\subsection{大義の条件}\label{ux5927ux7fa9ux306eux6761ux4ef6}

何かに対する敵対は、原理としての大義にはならない
大義は問いであり、顕現しない(顕現した原理は専制である)

\subsection{この章のまとめ}\label{ux3053ux306eux7ae0ux306eux307eux3068ux3081}
